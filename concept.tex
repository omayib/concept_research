\documentclass{beamer}
\usetheme{Madrid}
\usepackage{pifont}
\usepackage{amsmath}
\usepackage{geometry} 
\usepackage{svg}
\usepackage{graphicx}
\usepackage{tikz}

\graphicspath{ {./assets/} }
\usetikzlibrary{positioning}



\title{Adaptasi Positional Encoding pada Arsitektur Transformer untuk Sintesis Notasi Gamelan yang Koheren dan Terkendali}
\author{Arif Akbarul Huda}

\begin{document}
	\begin{frame}
		\titlepage
	\end{frame}
	
	\begin{frame}
		\frametitle{Previous Work}
		\framesubtitle{Decoding PDF files}
		
		% 1. TOP ROW: TWO COLUMNS (50% each)
		\begin{columns}
			% --- Left Column (Row 1, Col 1) ---
			\begin{column}{0.48\textwidth}
				\centering
				\includegraphics[width=0.9\linewidth]{notasi-ayak-nem-slendro.png}
				\footnotesize{Notasi gamelan Ayak-ayakan Nem Slendro pt. Nem}
			\end{column}
			
			% --- Right Column (Row 1, Col 2) ---
			\begin{column}{0.48\textwidth}
				\centering
				\includegraphics[width=0.9\linewidth]{cmap-font-balungan.png}
				\footnotesize{Cmap font balungan.}
			\end{column}
		\end{columns}
		
		% Optional: Add a small vertical space between the rows
		\vspace{0.5em}
		\hrule % Optional: A horizontal line to visually separate the rows
		
		% 2. BOTTOM ROW: ONE MERGED COLUMN
		% We use \vfill to push the content to the bottom and a minipage to control its width
		\vfill 
		\begin{minipage}{\textwidth} % Minipage spans the full text width
			\centering
			
			\begin{figure}
				\centering
				\includegraphics[width=0.75 \linewidth]{notasi-ayak-nem-slendro-decoded.png}
				\caption{Notasi ayakan decoded}			
			\end{figure}
		\end{minipage}
		
	\end{frame}
	
	\begin{frame}
		\frametitle{Previous Work}
		\framesubtitle{Plotting Structure}
		\begin{columns}
			
			% --- First Column (Left Image) srepeg-tlutur-sl-sanga.png---
			\begin{column}{0.5\textwidth}
				\begin{figure}
					\centering
					\includegraphics[width=1 \linewidth]{srepeg-tlutur-sl-sanga.png}
					\caption{Notasi Gamelan}			
				\end{figure}		
			\end{column}
			
			% --- Second Column (Right Image) ---
			\begin{column}{0.5\textwidth}
				\begin{figure}
					\centering
					\includegraphics[width=1 \linewidth]{plot-struktur-gamelan.png}
					\caption{Plot Struktur}			
				\end{figure}		
			\end{column}
		\end{columns}
	\end{frame}

	\begin{frame}
		\frametitle{Insight}
		\begin{figure}
			\centering
			\includegraphics[height=0.5 \linewidth]{book-fundamental-music-processing.png}
			\caption{Buku Referensi}			
		\end{figure}
	\end{frame}

	\begin{frame}
		\frametitle{Insight}
		\framesubtitle{Chapter 4. Music Structure Analysis}
		\begin{columns}
			
			% --- First Column (Left Image) srepeg-tlutur-sl-sanga.png--- structural-analysis.png
			\begin{column}{0.5 \textwidth}
				\begin{figure}
					\centering
					\includegraphics[width=1 \linewidth]{structural-analysis.png}
					\caption{4.1. From the book}
				\end{figure}
			\end{column}
			
			% --- Second Column (Right Image) ---
			\begin{column}{0.5 \textwidth}
				The General Goal of Music Structural Analysis
				\begin{itemize}
					\item  Temporal Segmentation
					\item  Structural Identification
					\item  Categorical Grouping
				\end{itemize}
				The methods
				\begin{itemize}
					\item  Repition-based
					\item  Novelty-based
					\item  Homogeneity-based
				\end{itemize}
				
			\end{column}
		\end{columns}
	\end{frame}
	\begin{frame}
		\frametitle{Insight}
		\framesubtitle{Chapter 4. Music Structure Analysis}
		\begin{columns}
			
			% --- First Column (Left Image) srepeg-tlutur-sl-sanga.png--- structural-analysis.png
			\begin{column}{0.5 \textwidth}
				\begin{figure}
					\centering
					\includegraphics[width=1 \linewidth]{structure-annotatoins-labeling-evaluation.png}
					\caption{4.30. From the book}
				\end{figure}
			\end{column}
			
			% --- Second Column (Right Image) ---
			\begin{column}{0.5 \textwidth}
				Evaluation
				\begin{itemize}
					\item  Precision, Recall, F-Measure
					\item  Structure Annotations
					\item  Labeling Eval.
					\item  Boundary Eval.
					\item  Thumbnail Eval.
				\end{itemize}
				
			\end{column}
		\end{columns}
	\end{frame}
	\begin{frame}
		\frametitle{Insight}
		
		
		Subjective evaluation are
		\begin{itemize}
			\item  Unscalable
			\item  Inability to Guide Improvement
			\item  Missing the "Why"
		\end{itemize}
	 	\vspace{1cm}
		\tiny de Berardinis, J., Cangelosi, A. and Coutinho, E. (2022) “Measuring the Structural Complexity of Music: From Structural Segmentations to the Automatic Evaluation of Models for Music Generation,” IEEE/ACM Transactions on Audio, Speech, and Language Processing, 30, pp. 1963–1976. Available at: https://doi.org/10.1109/TASLP.2022.3178203.
	\end{frame}

	
	
	\begin{frame}
		\frametitle{Insight}
		\tiny Kader FB, Karmaker S. A Survey on Evaluation Metrics for Music Generation [Internet]. arXiv; 2025 [cited 2025 Nov 19]. Available from: http://arxiv.org/abs/2509.00051
	
		\begin{figure}
			\centering
			\includegraphics[width=1 \linewidth]{taxonomy-evaluation-symbolic-music-generation}
			\caption{Cuplikan taxonomy evaluasi}			
		\end{figure}
	\end{frame}

	
	\begin{frame}
		\frametitle{P.O.C}
		\begin{itemize}
			\item Experimental evaluasi struktur pada dataset
			\item Experimental evaluasi struktur pada previous paper
		\end{itemize}
	\end{frame}
	
\end{document}
