\documentclass{beamer}
\usetheme{Madrid}
\usepackage{pifont}
\usepackage{amsmath}
\usepackage{geometry} 
\usepackage{svg}
\usepackage{graphicx} 

\graphicspath{ {./assets/} }




\title{Adaptasi Positional Encoding pada Arsitektur Transformer untuk Sintesis Notasi Gamelan yang Koheren dan Terkendali}
\author{Arif Akbarul Huda}

\begin{document}
	\begin{frame}
		\titlepage
	\end{frame}

	\begin{frame}
		\frametitle{Rumusan Masalah}
		Model LSTM pembangkit notasi gamelan yang dikembangkan oleh Fanani, A.Z. dkk. (2025) gagal menangkap relasi tersirat antarnotasi. Mekanisme pengacakan Algoritma Genetika (GA) digunakan Fanani, A.Z. dkk. (2025) untuk menangani kelemahan LSTM tersebut, tetapi GA justru berpotensi merusak struktur koherensi seluruh notasi. Kondisi tersebut mengakibatkan peran notasi terhadap keseluruhan struktur terabaikan sehingga ciri khas notasi dan identitas musikal gamelan memudar seiring bertambah panjangnya sekuens notasi. Apabila hal ini tidak diatasi, koherensi tematik dalam komposisi notasi gamelan tidak terwujud. Oleh karena itu diperlukan pendekatan baru, model pembangkit notasi gamelan yang dapat mempertimbangkan peran tersirat setiap notasi dalam struktur lagu melalui mekanisme perhatian (attention mechanism).
	\end{frame}
	\begin{frame}
		\frametitle{Original Result LSTM}
		\begin{figure}
			\centering
			\includegraphics[height=0.3 \linewidth]{original-lstm.png}
			\caption{Original Training Result Gamelan with LSTM}			
		\end{figure}		
	\end{frame}	
	\begin{frame}
		\frametitle{Reverse Engineering LSTM}
		\begin{figure}
			\centering
			\includegraphics[height=0.3 \linewidth]{reverse-code-40hl.png}
			\caption{Training Result Gamelan with LSTM}			
		\end{figure}		
	\end{frame}

	\begin{frame}
		\frametitle{Original Training Perform. Report LSTM}
		\begin{figure}
			\centering
			\includegraphics[height=0.3 \linewidth]{original-lstm-report-performance.png}
			\caption{Original Report LSTM}			
		\end{figure}		
	\end{frame}

	\begin{frame}
		\frametitle{Rebuild Training Perform. Report LSTM}
		\begin{figure}
			\centering
			\includegraphics[height=0.3 \linewidth]{rebuild-report-performance.png}
			\caption{Rebuild Report LSTM}			
		\end{figure}		
	\end{frame}
	\begin{frame}
		\frametitle{Progress}
		\begin{itemize}
			\item Koherensi Musikal
			\item Sequence Modeling
			\item User in the loop System
		\end{itemize}
	\end{frame}

	\begin{frame}
			\begin{figure}
			\centering
			\includegraphics[width=0.7\linewidth]{pitch-mismatch.png}
			\caption{Contoh pitch mismatch method}			
		\end{figure}
	
	Fu X, Deng H, Yuan X, Hu J. Generating High Coherence Monophonic Music Using Monte-Carlo Tree Search. IEEE Trans Multimedia. 2023;25:3763–72. 
	\end{frame}

	\begin{frame}
		\frametitle{P.O.C}
		\begin{itemize}
			\item Bagana membuktikan incoherence?
			\item Reverese Paper :  LSTM, BiLSTM, G.A. Small prev dataset Gamelan
			\item Objective : Function untuk scoring tingkat coherence pada gamelan
		\end{itemize}
	\end{frame}

	\begin{frame}
		
		\begin{figure}
			\centering
			\includegraphics[height=0.3 \linewidth]{syarif-lstm-gamelan.png}
			\caption{Rebuild LSTM}			
		\end{figure}
		(Input layer 19 , Hidden layer 200 units) Attempt more than 10x.
		Syarif AM, Azhari A, Suprapto S, Hastuti K. Gamelan Melody Generation Using LSTM Networks Controlled by Composition Meter Rules and Special Notes. JAIT [Internet]. 2023 [cited 2025 June 9]; Available from: http://www.jait.us/show-224-1287-1.html
	
	\end{frame}

	\begin{frame}
	\begin{figure}
		\centering
		\includegraphics[height=0.3 \linewidth]{lstm-gamelan.png}
		\caption{Rebuild LSTM with GA}			
	\end{figure}
	(Input layer 17 , Hidden layer 200 units)
	Fanani AZ, Maulana Syarif A, Novita Dewi I, Karim A. Enhancing Creativity and Validation in Explanatory Deep Learning-Based Symbolic Music Generation: A Hybrid Approach With LSTM and Genetic Algorithms. IEEE Access. 2025;13:105280–301. 
	\end{frame}
	
\end{document}
